Modern engines are faced with increasingly stringent requirements for reduced fuel consumption and lower emissions. A technique which can partly be used to reduce emissions of nitrogen oxides is recirculation of combusted gases (Exhaust Gas Recirculation, EGR). In gasoline engines, it also has the advantage that it can save fuel by reducing pumping losses. To large mixture of EGR in the air to the cylinders will however affect the combustion stability negatively. To investigate EGR rate and dynamics with respect to different actuator inputs, the thesis develops an engine model that includes EGR. The model focus on the air flow in the engine and extends an existing mean value engine model. Two types of EGR-system are investigated. They are short-route EGR which is implemented between intake manifold and exhaust manifold and long-route EGR which is implemented between compressor and turbine. The work provides a simulation study that compares both stationary and transient properties of the two EGR-systems, such as fuel consumption, maximum EGR, and rise time with respect to different actuators.
